\documentclass[a4paper]{article}
\usepackage[utf8]{inputenc}
\usepackage{fullpage}
\usepackage{csquotes}
\usepackage[ngerman]{babel}
\usepackage{biblatex}
\usepackage{float}
\usepackage{graphicx}
\usepackage{epstopdf}
\usepackage{subfigure}
\setcounter{secnumdepth}{-1} 
\usepackage{hyperref}
\usepackage{minted}
\usemintedstyle{friendly}
\bibliography{Dokumentation}
\title{ZooMTL}
\author{Gruppe Bazinga \\ Julian Heise, Matr.-Nr.: 782691 \\ Dominik Eckelmann, Matr.-Nr.: 785856 \\ Willi Schönborn, Matr.-Nr.: 774190}
\date{\today}
\begin{document}

\begin{figure}[H]
\centering
\includegraphics[width=0.5\textwidth]{beuth.eps}
\maketitle
\end{figure}

\section{Einleitung}
Als Teil der Lehrveranstaltung \textit{Software Engineering Vertiefung} im Wintersemester 2011/2012 an der \textit{Beuth Hochschule für Technik Berlin} sollte im Rahmen einer Übung mithilfe der Eclipse-Plattform ein simpler Model-To-Text-Transformator geschrieben werden. Das Ziel dieses Dokumentes ist es die Ergebnisse und Erkenntnise dieser Übung in Form eines Lernprotokolls zusammenzufassen.

\section{Erkenntnisse}
\begin{itemize}
  \item Die Ecore-Diagrammfunktion erscheint kompliziert und unausgereift. Stattdessen hat es sich als vorteilhaft erwiesen, die Baumstruktur für den Entwurf zu verwenden
  \item Nicht alle Eigenschaften von Ecore-Elementen werden für die Generierung in unserem Fall benötigt
  \item Auf den ersten Blick erinnert MTL an XSL und scheint auch ein ähnliches Einsatzgebiet zu haben. Es wäre wahrscheinlich auch möglich mit XSL Code zu generieren, wobei XML als Metamodell verwendet werden müsste
  \item Es besteht ein Konflikt zwischen der Formatierung der MTL-Datei und der Java-Datei: da alle Einrückungen und sonstige Formatierungen aus MTL in den Java-Code übernommen werden, entsteht nur schwer lesbarer Java Code. Andererseits könnte der Java-Code aufgeräumter sein, wobei wiederum die Darstellung des MTL leiden würde. In diesem Punkt gelang es uns nicht eine zufriedenstellende Lösung zu finden und gut wartbaren Code zu erzeugen
  \item OCL ist sehr mächtig und ermöglicht das effektive Behandeln von Mengen und Listen. Dabei lassen sich mit wenig Code gute Ergebnisse erzielen
  \item Wir vermissen die Möglichkeit das Ecore-Modell für die Ausgabe aufzubereiten. 
  \item Das Generieren von Java-Code erscheint sehr umständlich und teilweise redundant
  \item Es fehlt die Möglichkeit den MTL-Transformator zu parametriesieren
\end{itemize}

\section{Fragen}
Nachfolgende Fragen wurden während der Arbeit mit EMF aufgeworfen:
\begin{itemize}
  \item Gibt es eine Möglichkeit die Formatierung der MTL-Ausgabe in den Griff zu bekommen?
  \item Gibt es die Möglichkeit MTL Transformatoren zu parametrisieren?
\end{itemize}

\end{document}

