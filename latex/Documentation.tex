\documentclass[a4paper]{article}
\usepackage[utf8]{inputenc}
\usepackage{fullpage}
\usepackage{csquotes}
\usepackage[ngerman]{babel}
\usepackage{biblatex}
\usepackage{float}
\usepackage{graphicx}
\usepackage{epstopdf}
\usepackage{subfigure}
\setcounter{secnumdepth}{-1} 
\usepackage{hyperref}
\usepackage{minted}
\usemintedstyle{friendly}
\bibliography{Dokumentation}
\title{ZooMTL}
\author{Gruppe Bazinga \\ Julian Heise, Matr.Nr.: 782691 \\ Dominik Eckelmann, Matr.Nr.: 785856 \\ Willi Schönborn, Matr.Nr.: 774190}
\date{\today}
\begin{document}

\begin{figure}[H]
\centering
\includegraphics[width=0.5\textwidth]{beuth.eps}
\maketitle
\end{figure}

\tableofcontents

\newpage

\section{Einleitung}
Als Teil der Lehrveranstaltung \textit{Software Engineering Vertiefung} im Wintersemester 2011/2012 an der \textit{Beuth Hochschule für Technik Berlin} sollte im Rahmen einer Übung mithilfe der Eclipse-Plattform ein simpler Model-To-Text-Transformator geschrieben werden. Das Ziel dieses Dokumentes ist es die Ergebnisse und Erkenntnise dieser Übung in Form eines Lernprotokolls zusammenzufassen.

\section{Thema}
ECore-Modell (inkl/exkl Diagramm), Java-Code als Output, Generator in MTL, Zoo

\section{Lernprotokoll}
kein Diagramm, Modell-Editor (Baum-Editor) ist einfacher

viele Properties (unique, ...), man braucht nicht alle

MTL: ähnlich zu XML
Schwächen einer Template-Engine: alles generiert Output
relativ lesbare Aufrufe
OCL ist mächtig, FP style, fluent API

Strukturierter Quellcode als direkter Output ist umständlich: Imports etc verlangen, dass man viele Sachen doppelt macht
Sammeln von Informationen muss oft wiederholt werden

templates sind sinnvoll für kleine wiederkehrende Aufgaben (vgl. Funktionen/Methoden in normalen Programmiersprachen)
templates printen sofort raus, keine return values

lesbares Template vs lesbarer Quellcode, beides geht nicht

Parameterisierung von Templates? Formatierung, spezielle Annotations, etc

reusable MTL templates?

\addcontentsline{toc}{section}{Literatur}
\printbibliography

\addcontentsline{toc}{section}{Abbildungsverzeichnis}
\listoffigures

\addcontentsline{toc}{section}{Quellcodeverzeichnis}
\renewcommand\listoflistingscaption{Quellcodeverzeichnis}
\listoflistings

\end{document}

